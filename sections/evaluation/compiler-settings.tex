% Describe the test setup to verify that your problems from 1.3 have been solved.
% This can be done in different ways depending on focus of your problems.
% Some problems may purely objective, such as "improve the performance of X compared to Y".
% These are easy to evaluate since you simply need to compare the performance, and perhaps compare against a few more technologies that you have listed in Section 2 (related work).
% In other cases the problems may be very subjective, such as "Create a mobile app that can be used while driving, and which shows the most fuel efficient time to change gear".
% This problem will require a user-study in which several persons drive without the application, you calculate the fuel consumption, then they drive with the application and then you calculate the fuel consumption again.
% Then you collect the objective measurements (fuel consumption comparisons) and the subjective opinions from the users about whether the application was unobtrusive, usable, etc. (typically via a questionnaire).


The \emph{Rust} compiler doesne't give full acces to all the \emph{LLVM} settings 


Going all the possible options that can be sent into the \emph{rust} compiler there are two that are key for getting the most amount of debug information.
The most important one of these is the flag named \emph{debuginfo} which controls the amount of debug information generated.
It has three option, the first is option $0$ which means that no debug info will be generated.
The second option is $1$ which will only generate the line tables and the last is option $3$ which generates all the debug information it can.
The other key compiler option is the optimization flag named \emph{opt-level}, it controls the amount of optimization done to the code and which type of optimization, size or speed.
Comparing the different optimization levels the debug infomation got less and less when a higher optimization level was set on a simple blink code example.
The optimiztaion level $3$ resulted in amlost all variables being optimized out which made the debugging extreamly hard.
But if the optimiztaion level was set to $2$ there was in most cases enugh infomation to debug code.
Optimization level of $1$ hade the most debug information and was eisest to debug with.


\subsubsection{rustc settings}
\subimport{}{rustc.tex}


\subsubsection{LLVM settings}
\subimport{}{llvm.tex}

