% Explain the implementation of the vscode explanation.
Creating a \emph{VSCode} extension is simple because \emph{Microsoft} has a tool called \emph{TODO} \ref{TODO}, that will generate a empty extension.
They also have a lot of documentation on how to get started with creating a extension.
The documentation mentions how to create debug extensions.
There is two different types of debug extensions, the first is a extension that implements a debug adapter for a specific debugger.
These often require a lot work because a debug adapter needs to translate the \gls{dap} protocol messages to command that the debugger understands.
Thus, depending on the debugger it can require a lot of work to get this working well.
The other type of debug extension is just a wrapper for the debugger that already implements the \gls{dap} protocol.
These extension are very simple to create because the they only need to start the debugger and connect to it using the \emph{dap} protocol.
I refer the reader to \emph{Microsofts} documentations \ref{TODO} for further reading.

The debugger \emph{Embedded Rust Debugger} implements the \gls{dap} protocol over a \gls{tcp} server.
Thus, the \emph{VSCode} extension is just a wrapper that starts the debugger \gls{tcp} server and then connects to it.
Connecting to the debugger server over \gls{tcp} is very easy to do, because \emph{Microsoft} has a library called \emph{VSCode} that does that for you.
The extension also captures the logs of the debugger and outputs them to the user using the \emph{VSCode} library.

There are also some configurations that the user of the \emph{Embedded Rust Debugger} can set in the \emph{launch.json} file.
The available configurations are defined in a \emph{json} file called \emph{TODO} in the extension project.

