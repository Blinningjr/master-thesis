% This section describes the outcome of your work and summarizes your efforts.
% It also outlines things that are left to do to reach a full solution, or to integrate your solution with something else.

% The conclusion from the results
The result from testing and comparing the different settings in the compiler is a bit disappointing.
Because the settings found did not do much or were well known before, thus there was no real improvement.


But the result from comparing the debuggers was a success.
The main reason for this is that the debugger \gls{erd} is able to correctly evaluate the value of variables that are of the \emph{Rust} enum type.
While both of the two supported debugger by the \emph{Rust} team evaluated the wrong value in some special cases.
The debugger \gls{erd} was also able to simplify the process of debugging embedded systems that run \emph{Rust} code.


The debugging library \emph{rust-debug} was also successful in simplifying the process of retrieving debug information from \gls{DWARF}.
The key design feature that makes it very useful, is that it is designed to be platform independent.
Which makes the library a good contribution to the \emph{Rust} debugging community.


Overall we would say that goal of improving debugging of optimized \emph{Rust} code was achieved, but that there is still a lot more that needs to be done.


\section{Potential Future Debugging Improvement}
% Future work that can be done on the llvm
The ranges for the location information seam to be set a bit to tightly by \emph{LLVM} in some cases, meaning that the value still is in the debugged target for some time after the end address of the given address range.
There are three major factors that affects the variable location correctness, they are mentioned in section \ref{section:loc-ranges}.
Thus one potential improvement to debugging would be to improve how the variable location ranges are set.


\section{Future Debugger Improvement}
% Future work that can be done on the debuger side.
The debugger \gls{erd} only supports the most important functionalities of a debugger.
Thus there is many more features that could be added.
One important one is the ability to evaluate expressions which is left to be done as future work.
This is a hard problem because the evolution of the expressions should work exactly the same as the \emph{Rust} compiler.
Thus it would be best if the same code could be used so that it does not need to be written twice.


% Future work that can be done on the debuger side.
One of the main problems with debugging optimized code is that the variables are stored in registers and never pushed to the stack.
This causes the variables to be overwritten when they are not needed anymore.
That fact makes debugging very hard because the user has to stop the execution when the variable still exist.
But if the debugger is able to get the last value of the variable and store it.
The last known value of the variable could be displayed to the user if it is out of range.
There is no time to make a solution for this problem, thus this will have to be left for future work.


%\section{Future Debugger Library Improvement}
%% Future work that can be done on the debuger side.
%Currently the debugging library \emph{rust-debug} is not able to display values stored on the heap.
%It will instead show the pointer to the heap location where the value is stored.
%This problem is a little bit hard because the pointer is a data structure, thus it is hard to determine if it is a pointer or not.
%Implementing a solution for this is left as future work.


