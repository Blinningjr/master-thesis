% This section describes the outcome of your work and summarizes your efforts.
% It also outlines things that are left to do to reach a full solution, or to integrate your solution with something else.

% The conclusion from the results
TODO



\subsection{Future Debugging Improvement}
% Future work that can be done on the debuger side.
One of the main problems with debugging optimized code is that the variables are stored in registers and never pushed to the stack.
Thus after the variables are done with they are often overwritten and this makes it impossible for the debugger to know the value they had.
This makes debugging very hard because the user has to find the correct machine code locations where the value is in a registry, which is not easy.
But if the debugger is able to get the last value of the variable somehow it could display it as the last known value of the variable.
It could maybe be done by storing the value of the variable before it get overwritten, but that is a hard problem in it self.
There is no time to make a solution for this problem, thus this will have to be left for future work.


\subsection{Future Debugger Improvement}
% Future work that can be done on the debuger side.
The debugger presented in this thesis only supports most important functionalities of a debugger.
Thus there is many more features that could be added.
One important one is the ability to evaluate expression which is left to be done as future work.
This is a really hard problem because the evolution of the expressions should work exactly the same as the \emph{Rust} compiler.
Thus it would be best if the same code could be used so that it doesn't need to be written twice.


\subsection{Future Debugger Library Improvement}
% Future work that can be done on the debuger side.
Currently the debugging library \emph{rust-debug} is not able to display a pointer and the value it points to.
When a user wants to evaluate the value of a variable that is a pointer such as a string, they most often want the value of the string and not the pointer to the string.
This problem is a little hard because sometimes the user want the value of the pointer.
Implementing a solution for this is left as future work.


