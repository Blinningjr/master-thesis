A crucial part of programming has always been debugging the computer program.
Debugging is the process of finding and resolving errors, flaws, or faults in computer programs.
These errors, flaws, or faults are also commonly referred to as a bug or bugs in the field of computer science.
Debugging has become more challenging because of the increasing complexity of computer programs and hardware.
Therefore, better debugging tools have become essential to make the debugging process more accessible and time efficient.
It is especially true for debugging embedded systems because of their close relation to hardware.


One of the first debugging tools made, and one of the most valuable types, is a debugger.
A debugger is a program that allows the developer to control the debugged program in some ways, for example, continuing, halting, and resetting.
It can also inspect the debugged program by, for example, displaying the values of the variables.
Debuggers are especially useful when programming embedded systems because a debugger enables the developer to inspect what is happening in the \gls{mcu}.
It gives the developer a complete view of what the program is doing.
Debuggers are also valuable tools for testing.


Debugging today works by having the compilers generate debug information when compiling the program.
The debug information is stored in a file in a unique format designed to be read by a debugger or other debugging tools.
One of the most popular formats is named 
\gls{DWARF}, and it is a complex format that section \ref{sec:dwarf} explains.


The debug information stored in the \gls{DWARF} file can be used to locate the values of variables that, in most cases, are stored on the call stack in memory (section \ref{sec:callstack} explains what a call stack is).
The debug information also contains the data type information of the variable.
The debug information is used to interpret the raw value of a variable found in memory into a typed value.


Variables stored in registers are a significant problem with debugging optimized code.
Registers are faster to access than other memory and have a much lower capacity.
Therefore, storing variables there often increases the speed of the programs, but the variables are not persistent because they get overwritten.
It makes debugging a lot harder because the variables are very short-lived compared to variables located on the heap or call stack.
We refer the reader to section \ref{sec:regmem} for more detailed information about registers, the call stack, and the heap.


\section{Background}
In the \emph{Rust} programming language community, very few projects focus on debugging embedded systems and even less on improving the debugging of optimized \emph{Rust} code.
One of the larger projects focusing on debugging is named \emph{gimli-rs}.
One of the main goals of the projects is to make a \emph{Rust} library for reading the debugging format \gls{DWARF}.
Then there are other projects like \emph{probe-rs} that focus on making a library that provides different tools to interact with various \glspl{mcu} and debugging probes.
One of their newest tool, which is in early development, is a debugger that also uses the \emph{gimli-rs} library to read the DWARF file.


There is some work done on the \emph{LLVM} project to improve the generation of debugging information.
However, there is no clear focus on improving debugging for optimized code in the \emph{LLVM} project.


\section{Motivation}
A motivation for this thesis is that optimized \emph{Rust} code can be $10-100$ times faster than unoptimized code, according to the \emph{Rust} performance book \cite{perf-book}.
Compared to other languages, for example, \emph{C}, the optimized code can be about $1-3$ times faster than the unoptimized code, as seen in the paper \cite{clang-opt}.
Therefore, for a language like \emph{C}, it is much more acceptable not to be able to debug optimized code because the difference is not that large compared to \emph{Rust} code.
However, in the case of \emph{Rust}, the difference is significant.
Therefore, it is a problem because debuggers are bad at debugging optimized code, as explained in the introduction, and unoptimized code is too slow to run.
That is why there is a need for better debuggers that can provide a good experience when debugging optimized \emph{Rust} code.


An argument against the need to debug embedded systems is that the \emph{Rust} compiler will catch most of the errors.
It is especially true regarding pointers and memory access, which are the two most common causes of bugs in programming languages like \emph{C} and \emph{C++}.
Therefore, there is less need for a debugger that can debug \emph{Rust} code than there are debuggers that can debug \emph{C} and \emph{C++} code.
However, when it comes to embedded applications, there is still a need for debugging with such a debugger.
One reason is that \emph{Memory-mapped I/O} is used to perform \emph{input/output (I/O)} between the \emph{CPU} and peripheral devices.
Also, a debugger is an excellent tool for showing these peripherals' state, making the development and testing much more straightforward.
Another reason is that hardware bugs are much easier to find if it is possible to inspect the state of a device quickly.


Another motivation for creating a debugger for embedded systems is that the \emph{Rust} team's two supported \emph{Rust} debuggers are \emph{LLDB} (part of the \emph{LLVM} project) and \emph{GDB}, which at the time of writing, both require another program to interact with the \gls{mcu}.
An example of such a program is \emph{openocd} (openocd homepage \cite{openocd}), which works as a \emph{GDB} stub to which the debugger connects.
It complicates the process of debugging, especially for new developers.


Most debuggers for \emph{Rust} code are written in other programming languages, and there are not many debugging tools written in \emph{Rust} yet.
Therefore, one of the critical motivations for this thesis is to write a debugger in \emph{Rust}, which will also lead to the debugger having all the benefits of memory safety and the excellent ecosystem \emph{Rust} provides.


\section{Problem definition} \label{sec:problemDef}
This thesis's main problem is improving the experience of debugging optimized \emph{Rust} code on embedded systems.
It will be achieved by creating a new debugger and debugging library solving the three following problems:

\begin{itemize}
  \item Two of the most popular debuggers, \emph{GDB} and \emph{LLDB}, require another program to debug embedded systems.
This problem makes both of them harder to use and is a bad user experience.

\item Both \emph{GDB} and \emph{LLDB} often display that variables are optimized out when debugging optimized code, even with the less aggressive optimization options.
Their messages are very vague and can make it impossible to understand what is happening in the program.

\item Retrieving debugging information from the \emph{DWARF} format is challenging to do.
It makes creating tools that require debug information difficult and time-consuming to implement.
\end{itemize}

This thesis aims to create a new debugger and library that addresses the abovementioned problems.
The debugger should also support the most common debugging features listed in section \ref{sec:delimitations}.


\section{Delimitations}
\label{sec:delimitations}
Currently, there are a lot of different debugging features that other debuggers have.
Many of them are advanced and complicated to implement.
Therefore, the number of features is limited to the most important ones, limiting the scope of this thesis.
The following is the list of features the debugger is required to have:


\begin{itemize} \label{list:debuggerfeatures}
  \item Controlling the debugging target by:
  \begin{itemize}
    \item Continuing execution.
    \item Halting execution.
    \item Reset execution.
    \item Stepping
  \end{itemize}
  \item Set and remove breakpoints.
  \item Virtually unwind the call stack.
  \item Evaluating variables and their types.
  \item Finding source code location of functions and variables.
  \item A \acrfull{cli}.
  \item Support the \emph{Microsoft} \acrfull{dap}.
\end{itemize}


Debugging embedded systems requires that the debugger knows specific information about the target system, e.g., the architecture.
Supporting every \gls{mcu} out there would be too big of a scope for this thesis.
Therefore, the debugger must only work with the target system \emph{Nucleo-64, STM32F401RET6 64 pin}.


The compiler backend \emph{LLVM} that \emph{rustc} uses support two debugging file formats, according to their documentation \cite{llvm-dbs}.
One is the format \gls{DWARF}, and the other is \emph{CodeView} which \emph{Microsoft} develops.
To make a debugger that supports both formats would require much extra work that does not contribute to solving the main problems of this thesis.
Therefore, it has been decided only to support the \gls{DWARF} format because it has good documentation.


This thesis's scope does not include changing or adding to the \gls{DWARF} format.
The main reason is that it takes years for a new version of the standard to be released; therefore, this thesis cannot include such a change or addition.
Another reason is that it would take some time before the Rust compiler implements the new version.


The \gls{DWARF} format is very complex but well-documented.
Due to this, the description in section \ref{sec:dwarf} will not go into every detail of \gls{DWARF}.
Instead, it will focus on explaining the minimum theory to understand the implementation of the debugger.
For further details, we refer the reader to \cite{dwarf}.

\section{Sustainability}
% Technical sustainability refers to longevity of information, systems, and infrastructure and their adequate evolution with changing surrounding conditions.
Debuggers must be continuously developed to keep up with the evolving programming languages.
However, debugging formats like \gls{DWARF} specify the communication between the compiler and debugger.
Therefore, little maintenance is needed because \gls{DWARF} is infrequently updated.
This project tries to minimize this problem and be more sustainable by creating the debugging library \emph{rust-debug}.
The hope is that the library will gain a community around it, which can help evolve it and the debugger \gls{erdb} alongside the \emph{Rust} programming language.


Debugging embedded systems requires the debugger to have support for different kinds of \glspl{mcu}.
Therefore, support for new \glspl{mcu} must be added to the debugger, and the old ones maintained.
That is the primary technical sustainability problem that this project has.
The project minimizes this problem by using the open source library \emph{probe-rs}.
There are also some other libraries used to make the project more sustainable.


Overall technical sustainability is a significant problem that this project faces.
The only solution is that more people get interested in it and start to help out with maintaining and developing it.

