% Give the reader an overview of the area.
% Try to focus on the scientific area, but do not forget the red thread that binds the rest of the introduction together.

A key part of programming has always been debugging the computer program.
Debugging is the process of finding and resolving errors, flaws or faults in computer programs.
These errors, flaws or faults are also commonly referred to as a bug or bugs in the field of computer science.
Debugging has become more difficult to do over the years because of the increasing complexity of computer programs and the hardware.
Thus the importance of better debugging tools have become essential to make the process of debugging easier and more time efficient.


One of the first types of debugging tools made and one of the most useful is called a debugger.
A debugger is a program that allows the developer to control the debugged program in some ways, for example stopping, starting and resetting.
It is also able to inspect the debugged program by for example displaying the values of the variables.
Debuggers are a very useful tool for debugging but it is also a very useful tool for testing.


Debugging today works by having the compilers generate debug information when compiling the program.
The debug information is then stored in a file, the file is formatted using special file format designed to be read by a debugger or another debugging tool.
One of the most popular of these file formats is the format: \gls{DWARF}.
It is a complex format that is explained in detail in section \ref{sec:dwarf}.


The debug information stored in the \gls{DWARF} file can be used to find the location of variables, in most cases the value of variables are stored in memory on the call stack.
A stack is a data structure for storing information and is usually used to store all the variables for each function that is currently being executed.
Thus a debugger can use the debug information to find the location of a variable and then evaluate the value of it, which is then displayed to the user of the debugger.


The main problem with debugging optimized code is that variables are not stored on the call stack, which is in memory.
Instead they are temporarily stored in registers that are faster to access but cannot hold as much information as the memory.
This reduces the amount of memory needed and makes the program run faster.
It also makes debugging a lot harder because the variables are only present in the register for a very short time and thus the debugger will often not have access to that value.
Then there is the problem of debugger not being able to utilize all the available debug information.


\subsection{Background}
\subimport{introduction/}{background.tex}


\subsection{Motivation}
\subimport{introduction/}{motivation.tex}


\subsection{Problem definition}
\label{sec:problemdefinition}
\subimport{introduction/}{problem-definition.tex}


%\subsection{Equality and ethics}
%\subimport{introduction/}{equality-and-ethics.tex}

%\subsection{Sustainability}
%\subimport{introduction/}{sustainability.tex}


\subsection{Delimitations}
\subimport{introduction/}{delimitations.tex}


%\subsection{Thesis structure}
%\subimport{introduction/}{thesis-structure.tex}

