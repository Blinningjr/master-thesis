Debugging is an essential part of programming.
At this moment there is no debugger that is good at debugging optimized \emph{Rust} code.
It is a problem because unoptimized \emph{Rust} code is very slow compared to optimized.
The ability of debugging optimized code is especially important for embedded systems because of the close relation to hardware.
Thus a tool like a debugger is very useful because it enables the developer to see what is happening in the hardware when debugging embedded systems.


To improve debugging for optimized \emph{Rust} code this thesis will research the available compiler settings that affect the generation of debug information.
Also a debugger called \acrshort{erdb} written in \emph{Rust} will be presented and the goal with it is to improve debugging for embedded systems running \emph{Rust} code.
This thesis will explain the implementation of \acrshort{erdb} and the debug information format \acrshort{DWARF}.


The debugger \acrshort{erdb} will be compared to other well known \emph{Rust} debuggers to see if it improved on the experience of debugging optimized \emph{Rust} code on embedded systems.
The comparison shows that it did improve on debugging the \emph{Rust} enum type and temporarily optimized out values.
However there is still many improvements that can be made to debugging optimized \emph{Rust} code.

