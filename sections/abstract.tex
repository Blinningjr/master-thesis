Debugging is an essential part of programming.
At this moment there is no debugger that is good at debugging optimized \emph{Rust} code.
It is a problem because unoptimized \emph{Rust} code is very slow compared to optimized.
The ability of debugging optimized code is especially important for embedded systems because of the close relation to hardware.
Thus, a tool like a debugger is very useful because it enables the developer to see what is happening in the hardware when debugging embedded systems.


To improve debugging for optimized \emph{Rust} code, this thesis presents a debugger called \acrshort{erdb}.
And a debugging library called \emph{rust-debug}.
The goal of the library is to make it easier to make debugging tools, by making it easier to get debug information from the \gls{DWARF} format.


This thesis will explain the implementation of \acrshort{erdb} and \emph{rust-debug}, it will also explain how the debug information format \acrshort{DWARF} works.


The debugger \acrshort{erdb} will be compared to other debuggers to see if it achieved its goal.
Comparing the debuggers showed that the biggest problem with debugging optimized code is that there is not enough debug information produced by the compiler.

