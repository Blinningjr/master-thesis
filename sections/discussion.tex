% Discuss how you solved your problems from 1.3, and what the results were (from section 5).
% Describe alternative solutions, what you could have done differently, problems you encountered, how your results compare to other peoples results, etc.
% Go through each problem individually, and then in the end add general remarks and discussion points which are outside the problems themselves but that you think may be valuable to share with the reader.
% This section can have several subsections.


% Overall Discussion of how the problem is solved

\section{Usefulness of \emph{rust-debug}}
The \emph{rust-debug} library makes it much easier to get information from \gls{DWARF}.
One reason is that it requires much less knowledge about \gls{DWARF}.
Also, the documentation for \emph{rust-debug} is much less text than the specification for the \gls{DWARF} format.
Therefore, reading and understanding take less time, making it easier to use.


The amount of code needed to get debug information is much less using \emph{rust-debug}, as section \ref{sec:evalrd} shows.
Using the number of code lines to measure complexity is not the most accurate method.
However, when the difference in the result is this significant, it shows that \emph{rust-debug} is less complex to use.



\section{Debug Experience For Optimized Rust Code}
At the beginning of this thesis, it was thought that \emph{GDB} could not evaluate variables located in registers.
This thought came from the observation that \emph{GDB} sometimes show all the variables as optimized out when debugging optimized code.
That seemed impossible in some cases because then the code should not work.
However, \emph{GDB} was right in that the variables were wholly optimized out.


With the expectation that those variables could still be debugged in mind, the result in section 5.3 is disappointing.


Unexpectedly, both \emph{GDB} and \emph{LLDB} have other minor problems.
Such as, \emph{GDB} parsed $64$-bit integers as $32$-bit.
Those minor problems might be specific to the debugger's \gls{cli} and therefore not present in other releases of the debuggers.
However, the problems show that \gls{erdb} has some slight advantages.



\subsection{Debugging Rust Code on Embedded Systems}
The experience of debugging \emph{Rust} code on embedded systems with both \emph{LLDB} and \emph{GDB} could be better.
It requires much configuring and a program like \emph{openocd}, which handles the communication between the debugger and the target device.
Using \gls{erdb} is easier because it removes that step.



\section{Accuracy of the DWARF Location Ranges}\label{section:loc-ranges}
While working on \emph{rust-debug} and \gls{erdb}, it was discovered that variables could sometimes live longer on the debugged target than described in \gls{DWARF}.
Therefore, some variables can potentially be read when \gls{DWARF} says they have no location.


The documentation for \emph{LLVM} \cite{llvm-dbs} mentions three significant factors that affect the variable location fidelity, they are:

\begin{enumerate}
  \item Instruction Selection
  \item Register allocation
  \item Block layout
\end{enumerate}

The documentation about the three significant factors does not mention how the ranges could be set too tight.
Therefore, to find out if this can be improved, there has to be a good understanding of \emph{LLVM}, which is out of the scope of this thesis.


The documentation also mentions that \emph{LLVM} ignores some location changes in some situations.
One situation they mention is that location changes are ignored for a function's prologue and epilogue code.
However, this should not be a problem because most debuggers will step over the prologue and epilogue code.

