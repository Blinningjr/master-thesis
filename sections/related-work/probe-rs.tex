% Describe existing solutions and other peoples scientific work in the same area.
% Note that the related work should focus around things that relate to your problems in 1.3.
% Example: if you are working on a social networking component which is to aggregate real-time data to automatically update your status in the social network, then your related work is not about other social networks but about different aggregation techniques, activity recognition solutions, etc.

The \emph{probe-rs} project is currently creating a debugger in \emph{Rust}, which is almost finished.
Their debugger is made for debugging embedded systems.
It uses their other tools in their library to access and control the debug target.
They also use the \emph{gimli-rs} library for reading the \gls{DWARF} format.


The main difference between \emph{probe-rs} and the debugging library \emph{rust-debug} presented in section \ref{subsection:rust-debug} of this thesis, is that \emph{rust-debug} is designed to be platform and architecture independent.
The \emph{probe-rs} library is designed to provide tools for embedded \glspl{mcu} and debug probes only.
While \emph{rust-debug} is designed to provide a layer of abstraction over the \gls{DWARF} debug format, which simplifies the process of retrieving debug information from \emph{DWARF}.

The main pros of \emph{rust-debug} compared to \emph{probe-rs} are:
\begin{itemize}
  \item Platform and architecture independence.
  \item Has a wider range of use cases.
  \item Less dependencies.
\end{itemize}

The main cons of \emph{rust-debug} compared to \emph{probe-rs} are:
\begin{itemize}
  \item A little bit more complex to use.
  \item Requires that the user provide the functionality of reading the debug target's memory and registers.
  \item The code is a bit more complex.
\end{itemize}

