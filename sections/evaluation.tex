% Describe the test setup to verify that your problems from 1.3 have been solved.
% This can be done in different ways depending on focus of your problems.
% Some problems may purely objective, such as "improve the performance of X compared to Y".
% These are easy to evaluate since you simply need to compare the performance, and perhaps compare against a few more technologies that you have listed in Section 2 (related work).
% In other cases the problems may be very subjective, such as "Create a mobile app that can be used while driving, and which shows the most fuel efficient time to change gear".
% This problem will require a user-study in which several persons drive without the application, you calculate the fuel consumption, then they drive with the application and then you calculate the fuel consumption again.
% Then you collect the objective measurements (fuel consumption comparisons) and the subjective opinions from the users about whether the application was unobtrusive, usable, etc. (typically via a questionnaire).

\section{Evaluating \emph{rust-debug}} \label{sec:evalrd}
The debugging library \emph{rust-debug} aims to solve the problem of getting debug information from the \gls{DWARF} format.
The difficult part is that it requires many lines of code and knowledge about the \gls{DWARF} format, to get the wanted debug information.
One way to measure if \emph{rust-debug} makes it easier to retrieve debug information, is to compare the amount of code lines it takes to create a debugger with and without using \emph{rust-debug}.


Unfortunately, there is no other debugger that has the exact same features as \gls{erdb}.
Also, it would be unfair to compare \gls{erdb} with a debugger like \emph{GDB}, which has a lot more features.


Instead, the number of lines in \emph{rust-debug} will be compared against the debugger module in \gls{erdb}.
The number of \emph{Rust} code lines was counted using the tool \emph{tokei} \cite{tokei}.
\emph{rust-debug} contains $3626$ lines of \emph{Rust} code, and the debugger module contains $1358$ lines.
Also, the whole debugger \gls{erdb} contains $3068$ lines of \emph{Rust} code.


The amount of \emph{Rust} code in \emph{rust-debug} is more than double that of the debugger module in \gls{erdb}.
Thus, \emph{rust-debug} has made it significantly easier to get some debug information.



\section{Evaluating \emph{ERDB}} % TODO
% List all the features it has.
One of the requirements set in the section \ref{sec:problemDef} was that \gls{erdb} needs to implement some of the most common debugging features.
A list of these features can be found in section \ref{sec:delimitations}.
\gls{erdb} implements all of those mentioned features, but there are two of them that are not fully supported.


% stepping, sw breakpoints, 
Currently, \gls{erdb} only supports hardware breakpoints, creating a limit to how many breakpoints can be set.
To have more breakpoints the debugger needs to support software breakpoints, which will make it possible to have a lot more breakpoints.
The other feature that is not fully supported are the different stepping variants.
Stepping one machine code instruction is currently the only supported stepping function.
It would have been very useful if it supported stepping one source code instruction.


% Dose not require external program for embedded systems.
One of the main goals of \gls{erdb} was to improve the user experience, by mainly removing the hassle of using an external program to debug embedded systems.
\gls{erdb} has achieved this goal by using the \emph{probe-rs} library.
\emph{probe-rs} allows \gls{erdb} to access the debug target without starting another program, which greatly improves the user experience.
There is also a \emph{vscode} extension for \gls{erdb} that makes the user experience as good as other debuggers.



\section{Debugger Comparison} % TODO
\label{sec:debuggercomparison}
% TODO: Compare to erdb, gdb and lldb.

% TODO: Mention the testing of the three debuggers.

The debugger \gls{erdb} needs to be compared to the already existing debuggers to see if any improvement is made to debugging optimized \emph{Rust} code.
The two most popular debugger used for debugging \emph{Rust} code are \emph{GDB} and \emph{LLDB}.
They are also the two debuggers that are supported by the \emph{Rust} development team according to the \emph{rustc-dev-guide} \cite{rust-dev-guide}.
%The criteria they will be compared to is how well the debuggers can retrieve debug information from optimized code and how well they can display that information to the user.
%Because getting the information is important but it is pretty much useless if it is not displayed in an user friendly way.


The testing and comparison of the three different debuggers is done manually on some example code, see the git repository \cite{example-code} for the example code.
The example code was modified to test how well the three debugger handled different situations.

To keep the comparison fair a breakpoint was added to the same machine code instructions when comparing the debugger.
This ensures that all the three debuggers stop on the same instruction.
Also, the latest released version of each compiler was used to keep the comparison fair.
At the time of writing the latest versions of debuggers are:

\begin{itemize}
    \item \emph{GDB} 12.1
    \item \emph{LLDB} 14.0.6
    \item \gls{erdb} 0.2.0 % TODO: Do a release
\end{itemize}




\subsection{Unoptimized Code Comparison}
All the debuggers were first tested on unoptimized \emph{Rust} code, to see that the debuggers were correctly installed and configured.
While testing them there were some differences between the debuggers that are interesting.
The most significant one being that \emph{LLDB} is not able to evaluate \emph{Rust} \emph{enums}, which both \emph{GDB} and \gls{erdb} are capable of.


Another difference found is that both \emph{LLDB} and \emph{GDB} evaluated a \emph{f32} to $10.1999998$, and \gls{erdb} evaluated it to $10.2$.
In the \emph{Rust} source code, the \emph{f32} was assigned the value of $10.2$, and all the debuggers returned that the raw bytes were $0x33332341$ in hexadecimal.
Reading those raw bytes as a $32$-bit float using little endian, gives $10.2$.



\subsection{Optimized Code Comparison} % TODO:
Debugging the code with optimization $2$, \emph{LLDB} differed a bit from the other two debuggers.
It is still not able to evaluate \emph{Rust enums}.
Also, it wrote all $8$-bit integers in hexadecimal format, but the values were correct.


\emph{GDB} differed in that it evaluated all the $64$ bit integers as $32$ bit integers, thus it showed the wrong values.
It did not have this problem when debugging the same code but unoptimized.


Compared to the other debuggers, \gls{erdb} did not have any problems with evaluating variables.
However, it sometimes does not give the correct stack trace when there are inline functions.
This problem has only occurred on optimized code.


All the three debuggers where equally bad when it comes to the main problem of debugging optimized code, which is the problem of many variables being optimized out.
That in turn makes it very hard to understand what is happening in the program, and thus hard to debug.

