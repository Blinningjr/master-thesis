% Describe the test setup to verify that your problems from 1.3 have been solved.
% This can be done in different ways depending on focus of your problems.
% Some problems may purely objective, such as "improve the performance of X compared to Y".
% These are easy to evaluate since you simply need to compare the performance, and perhaps compare against a few more technologies that you have listed in Section 2 (related work).
% In other cases the problems may be very subjective, such as "Create a mobile app that can be used while driving, and which shows the most fuel efficient time to change gear".
% This problem will require a user-study in which several persons drive without the application, you calculate the fuel consumption, then they drive with the application and then you calculate the fuel consumption again.
% Then you collect the objective measurements (fuel consumption comparisons) and the subjective opinions from the users about whether the application was unobtrusive, usable, etc. (typically via a questionnaire).

% TODO: Remove
There are three problems that this thesis tries to solve, they are mentioned in section \ref{sec:problemDef}.
The three following section will go thought how each of the problems were evaluate and what the results are.





\section{Evaluating \emph{rust-debug}}
The debugging library \emph{rust-debug} aims to solve the problem of it being difficult to get debug information from the \gls{DWARF} format.
Where the difficult part is that it requires many lines of code and knowledge about the \gls{DWARF} format, to get the debug information wanted.
Thus, one way to measure if \emph{rust-debug} makes it easier to retrieve debug information, is to compare the amount of code lines it takes to create a debugger with and without using \emph{rust-debug}.


Unfortunately there is no other debugger that has the exact same features as \gls{erdb}.
And it would be unfair to compare \gls{erdb} with a debugger like \emph{GDB}, which has a lot more features.


Instead the number of lines in \emph{rust-debug} will be compared against the debugger module in \gls{erdb}.
The number of \emph{Rust} code lines will be counted using the tool \emph{tokei} \cite{tokei}.
\emph{rust-debug} contains $3550$ lines of \emph{Rust} code, and the debugger module contains $1310$ lines.
Also, the whole debugger \gls{erdb} contains $2834$ lines of \emph{Rust} code.


The amount of \emph{Rust} code in \emph{rust-debug} is more then double that of the debugger module in \gls{erdb}.
Thus, \emph{rust-debug} has made it significantly easier to get some debug information.


\section{Evaluating \emph{ERDB}} % TODO
\section{Debugging Optimized Out Variables} % TODO
