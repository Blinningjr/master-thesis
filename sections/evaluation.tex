% Describe the test setup to verify that your problems from 1.3 have been solved.
% This can be done in different ways depending on focus of your problems.
% Some problems may purely objective, such as "improve the performance of X compared to Y".
% These are easy to evaluate since you simply need to compare the performance, and perhaps compare against a few more technologies that you have listed in Section 2 (related work).
% In other cases the problems may be very subjective, such as "Create a mobile app that can be used while driving, and which shows the most fuel efficient time to change gear".
% This problem will require a user-study in which several persons drive without the application, you calculate the fuel consumption, then they drive with the application and then you calculate the fuel consumption again.
% Then you collect the objective measurements (fuel consumption comparisons) and the subjective opinions from the users about whether the application was unobtrusive, usable, etc. (typically via a questionnaire).

% TODO: Remove
There are three problems that this thesis tries to solve, they are mentioned in section \ref{sec:problemDef}.
The three following section will go thought how each of the problems were evaluate and what the results are.


\section{Evaluating \emph{rust-debug}}
The debugging library \emph{rust-debug} aims to solve the problem of it being difficult to get debug information from the \gls{DWARF} format.
Where the difficult part is that it requires many lines of code and knowledge about the \gls{DWARF} format, to get the debug information wanted.
Thus, one way to measure if \emph{rust-debug} makes it easier to retrieve debug information, is to compare the amount of code lines it takes to create a debugger with and without using \emph{rust-debug}.


Unfortunately there is no other debugger that has the exact same features as \gls{erdb}.
And it would be unfair to compare \gls{erdb} with a debugger like \emph{GDB}, which has a lot more features.


Instead the number of lines in \emph{rust-debug} will be compared against the debugger module in \gls{erdb}.
The number of \emph{Rust} code lines will be counted using the tool \emph{tokei} \cite{tokei}.
\emph{rust-debug} contains $3550$ lines of \emph{Rust} code, and the debugger module contains $1310$ lines.
Also, the whole debugger \gls{erdb} contains $2834$ lines of \emph{Rust} code.


The amount of \emph{Rust} code in \emph{rust-debug} is more then double that of the debugger module in \gls{erdb}.
Thus, \emph{rust-debug} has made it significantly easier to get some debug information.



\section{Evaluating \emph{ERDB}} % TODO
% List all the features it has.
One of the requirements set in the section \ref{sec:problemDef} was that \gls{erdb} needs to implement some of the most common debugging features.
A list of these features is can be found in section \ref{sec:delimitations}.
\gls{erdb} implements all of those mentioned features, but there are two of them that is not fully supported.


% stepping, sw breakpoints, 
Currently \gls{erdb} only supports hardware breakpoints, thus there is a limit to how many breakpoints can be set.
To have more breakpoints the debugger needs to support software breakpoints, which will make it possible to have a lot more breakpoints.
The other feature that is not fully supported are the different stepping variants.
Stepping one machine code instruction is currently the only supported stepping function.
But, it would have been very useful if it supported stepping one source code instruction.


% Dose not require external program for embedded systems.
One of the main goals of \gls{erdb} was to improve the user experience, by mainly removing the hassle of using a external program to debug embedded systems.
\gls{erdb} has achieved this goal by using the \emph{probe-rs} library.
\emph{probe-rs} allows \gls{erdb} to access the debug target without starting another program, which greatly improves the user experience.
There is also a \emph{vscode} extension for \gls{erdb} that makes the user experience equally as good as other debuggers.



\section{Debugger Comparison On Optimized code} % TODO
\label{sec:debuggercomparison}
% TODO: Compare to erdb, gdb and lldb.
The debugger \gls{erdb} needs to be compared to the already existing debuggers to see if any improvement is made to debugging optimized \emph{Rust} code.
The two most popular debugger used for debugging \emph{Rust} code are \emph{GDB} and \emph{LLDB}.
They are also the two debuggers that are supported by the \emph{Rust} development team according to there \emph{rustc-dev-guide} \cite{rust-dev-guide}.
The criteria they will be compared to is how well the debuggers can retrieve debug information from optimized code and how well they can display that information to the user.
Because getting the information is important but it is pretty much useless if it is not displayed in an user friendly way.


The testing and comparison of the three different debuggers is done manually on some example code, see the git repository \cite{example-code} for the example code.
The example code was many times modified to test how well the three debugger handled the different situations.
This was repeated until one of the debuggers gave an unexpected result.
There was three of these cases found where the code was compiled with optimization 2.
To keep the comparison fair a breakpoint was added to the same machine code instructions when comparing the debugger.
This ensures that all the three debuggers stop on the same instruction.



\subsection{Evaluation of \emph{Rust} Enums}
When testing the three different debuggers it was discovered that they evaluated \emph{Rust} enums to different values in a certain situation.
That situation happens when the value describing the variant of an enum is optimized out by the compiler.
Table \ref{table:enum3} shows an example of this situation, by showing the result of the three debuggers.

\begin{table}[h]
	\centering
	\small
	\begin{tabular}{ |p{2cm}|p{8cm}|  }
		\hline
		\multicolumn{2}{|c|}{\textbf{The debuggers evaluation results for variable \emph{test\_enum3}}} \\ 
		\hline
		\hline
		Source & Value \\
		\hline
%		\acrfull{pc} & 0x08000464 \\

		DWARF Location & (DW\_OP\_piece: 9; DW\_OP\_breg13 (r13): 156; DW\_OP\_piece: 3) \\

		Rust Source Code & let mut test\_enum3 = TestEnum::Struct(TestStruct \{ flag: true, num: 123\}); \\
		\hline
		\hline
		Debugger (Version) & Evaluated Result \\
		\hline
		ERD & test\_enum3 = 
		TestEnum \{ \textless \ OptimizedOut \textgreater \ \}\\

		GDB (12.1)  & (gdb) p test\_enum3\newline
		\$ 1 = nucleo\_rtic\_blinking\_led::TestEnum::ITest(\newline
		\textless optimized out\textgreater) \\

		LLDB (14.0.6) & (nucleo\_rtic\_blinking\_led::TestEnum) test\_enum3 = \{\newline
		\text{\ \ ITest = (0 = 0)}\newline
		\text{\ \ UTest = (0 = 0)}\newline
		\text{\ \ Struct = \{}\newline
		\text{\ \ \ \ 0 = (flag = false, num = 0)}\newline
		\text{\ \ \}}\newline
		\text{\ \ Non = \{\}}\newline
		\} \\
		\hline
	\end{tabular}
	\caption{The different debuggers evaluation result for variable \emph{test\_enum3}, and the actual source code value and DWARF location.}
	\label{table:enum3}
\end{table}


The declared value of the variable \emph{test\_enum3} can be seen in the row called \emph{Rust Source Code} in table \ref{table:enum3}.
Note that the variant of the enum is called \emph{Struct}, the value of flag is \emph{true} and the value of \emph{num} is $123$.


The location of the variables value can be seen in the row called \emph{DWARF Location} in table \ref{table:enum3}, there it can be seen that the first \gls{DWARF} operation is \emph{DW\_OP\_piece: 9}.
This operation in short means that the first $9$ bytes of the variable is optimized out, because there are no other operation before it.
The other two operations describe that the last $3$ bytes of the value is stored in memory at the address stored in the base register $13$ plus the offset $156$.
This all means that most of \emph{test\_enum3} is optimized out, including the value describing the enum variant.


The result from evaluating \emph{test\_enum3} using the debugger \gls{erdb} is that the whole enum is optimized out, as can be seen in table \ref{table:enum3}.
Thus, \gls{erdb} gives the correct result in this case.
But looking at the table it can be seen that both \emph{GDB} and \emph{LLDB} give different results.
\emph{GDB} give the incorrect result in that it says that the variant of the enum is \emph{ITest} when it should be \emph{Struct}.
\emph{LLDB} does not give any indication of what enum variant the value is, instead it shows the value of each enum variation.
But it gives the wrong value for attributes \emph{flag} and \emph{num} in the variant \emph{Struct}, thus it gives an incorrect result.
Thus out of the three debuggers it is \gls{erdb} that gives the most correct result, in these situations.



\subsection{Evaluation of \emph{Rust} structs}
When testing the three different debuggers it was discovered that they evaluated \emph{Rust} structs to different values in a certain situation.
The table \ref{table:struct} shows an example of this situation, by showing the result of the three debuggers.


\begin{table}[h]
	\centering
	\small
	\begin{tabular}{ |p{2cm}|p{8cm}|  }
		\hline
		\multicolumn{2}{|c|}{\textbf{The debuggers evaluation results for variable \emph{test\_struct}}} \\ 
		\hline
		\hline
		Source & Value \\
		\hline
%		\acrfull{pc} & 0x800059c \\

		DWARF Location & (DW\_OP\_breg13 (r13): 48; DW\_OP\_piece: 4) \\

		Rust Source Code & let mut test\_struct = TestStruct \{ flag: true, num: 123 \}; \\
		\hline
		\hline
		Debugger (Version) & Evaluated Result \\
		\hline
		ERD & test\_struct = TestStruct \{ num::123, flag::\textless \ OptimizedOut \textgreater \} \\

		GDB (12.1)  & (gdb) p test\_struct\newline
		\$ 1 = nucleo\_rtic\_blinking\_led::TestStruct \{flag: \textless synthetic pointer\textgreater, num: 123\} \\

		LLDB (14.0.6) & (nucleo\_rtic\_blinking\_led::TestEnum) test\_struct = (flag = false, num = 123) \\
		\hline
	\end{tabular}
	\caption{The different debuggers evaluation result for variable \emph{test\_struct}, and the actual source code value and DWARF location.}
	\label{table:struct}
\end{table}


The declared value of the variable \emph{test\_struct} can be seen in the row called \emph{Rust Source Code} in table \ref{table:struct}.
Note that the value of flag is \emph{true} and that the value of \emph{num} is $123$.


The location of the variables value can be seen in the row called \emph{DWARF Location} in table \ref{table:struct}, there it can be seen that it only has two \gls{DWARF} operation.
The \gls{DWARF} operation \emph{DW\_OP\_breg13: 260} means that the value is stored in memory, at the address stored in the base register $13$ plus the offset $260$. 
And the second operation \emph{DW\_OP\_piece: 8} means that the size of the value stored in the address of the previous operation is $4$ bytes.
The variable \emph{test\_struct} is larger than $4$ bytes, which in this case means that the attribute \emph{flag} is optimized out.


The result from evaluating \emph{test\_struct} using the debugger \gls{erdb} is that the attribute \emph{num} has the value $123$ and the attribute \emph{flag} is optimized out, as can be seen in table \ref{table:struct}.
Thus, \gls{erdb} is able to evaluate the correct value, but \emph{GDB} gives an even better result.
Because \emph{GDB} points out that the attribute \emph{flag} is a synthetic pointer.
A synthetic pointer is a pointer that is never dereferenced, and thus the actual value is not needed. 
Lastly the debugger \emph{LLDB} says that the value of \emph{flag} is \emph{false} which is incorrect.
Thus out of the three debugger \emph{GDB} is the most descriptive and correct, in these situations.



\subsection{Displaying Optimized Out Variables}
There is one more situation where the result of the debuggers are a bit different.
That situation is when a variable is optimized out, but only temporarily.
The table \ref{table:u16} shows an example of this situation, by showing the result of the three debuggers.


\begin{table}[h]
	\centering
	\small
	\begin{tabular}{ |p{2cm}|p{8cm}|  }
		\hline
		\multicolumn{2}{|c|}{\textbf{The debuggers evaluation results for variable \emph{test\_u16}}} \\ 
		\hline
		\hline
		Source & Value \\
		\hline
		\acrfull{pc} & 0x08001290 \\

		DWARF Location List Ranges & 0xffffffff 	0x080002f4\newline
    					     0x080004a4 	0x080004d0\\

		Rust Source Code & let mut test\_u16: u16 = 500; \\
		\hline
		\hline
		Debugger (Version) & Evaluated Result \\
		\hline
		ERD & test\_u16 = \textless OutOfRange\textgreater \\

		GDB (12.1)  & (gdb) p test\_u16\newline
		\$ 1 = \textless optimized out\textgreater \\

		LLDB (14.0.6) & (unsigned short) test\_u16 = \textless variable not available\textgreater \\
		\hline
	\end{tabular}
	\caption{The different debuggers evaluation result for variable \emph{test\_u16}, and the actual source code value and DWARF location.}
	\label{table:u16}
\end{table}


The declared value of the variable \emph{test\_u16} can be seen in the row called \emph{Rust Source Code} in table \ref{table:u16}.
The location of the variable is not described in the table because it has none.
Instead, the machine code instruction ranges for the location list entries are shown, they are found in the row called \emph{DWARF Location List Ranges}.
The first column of the ranges shows the start address of the range and the second column shows the end address of the range.
The first  entry has a larger start address because the first entry has special rules.
This means that there is only one location list entry in this case.


The \gls{pc} value in the table \ref{table:u16} is $0x08001290$, which is not between the range $0x080002f4$ to $0x080004d0$.
This means that \emph{test\_u16} will or has had a value, but correctly it does not have one.


The result of evaluating \emph{test\_u16} using \gls{erdb} is \emph{\textless OutOfRange\textgreater}, as can be seen in table \ref{table:u16}.
That is a unique message that is only used when the \gls{pc} value is not in range of any location list entry.
The debugger \emph{LLDB} gives the same result, except that the message is worded a bit differently.
However, the result from \emph{GDB} is not as good, because it is less descriptive.
Thus, both \emph{LLDB} and \gls{erdb} are more verbose and descriptive than \emph{GDB} in these situations.
