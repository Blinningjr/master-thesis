% Describe the test setup to verify that your problems from 1.3 have been solved.
% This can be done in different ways depending on focus of your problems.
% Some problems may purely objective, such as "improve the performance of X compared to Y".
% These are easy to evaluate since you simply need to compare the performance, and perhaps compare against a few more technologies that you have listed in Section 2 (related work).
% In other cases the problems may be very subjective, such as "Create a mobile app that can be used while driving, and which shows the most fuel efficient time to change gear".
% This problem will require a user-study in which several persons drive without the application, you calculate the fuel consumption, then they drive with the application and then you calculate the fuel consumption again.
% Then you collect the objective measurements (fuel consumption comparisons) and the subjective opinions from the users about whether the application was unobtrusive, usable, etc. (typically via a questionnaire).


To evaluate the soultion to the problem(see section \ref{sec:problemdefinition}) there is three criterias.
The first one is how much more useful debug infroatmion does the solution get from the compiler, this is to evalaute the first part of the problem. 
For the secound part of the probelm the debugger \gls{erd} needs to be compared to the aleady existing debuggers to see if any improvment is made to debugging optimized \emph{Rust} code.
The two most popular debugger used for debugging \emph{Rust} code are \emph{GDB} and \emph{LLDB}, thus it is these two debuggers that will be compared to the presented debugger.
The reason being that they are the two debuggers that are supported by the \emph{Rust} dev team acording to there \emph{rustc-dev-guide} \cite{rust-dev-guide}.
The cirteria they will be compared to is how well the debuggers can retrive debug infromation from optimized code and how well they can display that information to the user.
Because getting the information is important but it is pretty much useless if it is not displayid in a user friendly way.


\subsection{Compiler settings comparison}
\label{sec:settingscomparison}
\subimport{evaluation/}{compiler-settings.tex}


\subsection{Debugger Comparison}
\label{sec:debuggercomparison}
\subimport{evaluation/}{comparison.tex}

