% Primarily this section should be about scientific methods and theories you need to evaluate/compare/invent to solve your problems from 1.3.
% In some cases it may be ok to describe different technologies, but the purpose is to describe something and then draw a conclusion from that.
% Example, if you decide to discuss different databases, it may be for the purpose of selecting the best type for your implementation later on (based on for example data representation, scalability, speed, etc.).
% Optimally the problems in 1.3 are not solved by anyone else yet, in which case this section needs to describe how to solve them (new algorithms, mathematical approaches, etc.).
 
% This section can have a lot of subsections (3.1, 3.2, 3.3, etc).

% TODO: Explain DWARF Attribute

The information in the \glspl{die} are stored in attributes these attributes consists of a attribute name and a value.
The name of the attribute is used to know what the value the attribute hold should be used for, it is also used to differentiate the different attributes.
All of the attributes names start with \emph{DW\_AT\_} and then some name that describes the attribute, an example is the name attribute \emph{DW\_AT\_name}.
In the dwarf file the name of the attributes will be abbreviated to there abbreviation number that can be decoded using the \emph{.debug\_abbrev} section.
A die can only have one of each attributes and it thus limited to the number of attributes it can have.

