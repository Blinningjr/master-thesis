% Primarily this section should be about scientific methods and theories you need to evaluate/compare/invent to solve your problems from 1.3.
% In some cases it may be ok to describe different technologies, but the purpose is to describe something and then draw a conclusion from that.
% Example, if you decide to discuss different databases, it may be for the purpose of selecting the best type for your implementation later on (based on for example data representation, scalability, speed, etc.).
% Optimally the problems in 1.3 are not solved by anyone else yet, in which case this section needs to describe how to solve them (new algorithms, mathematical approaches, etc.).
 
% This section can have a lot of subsections (3.1, 3.2, 3.3, etc).

% TODO: Explain Unwinding call stack

To virtualy unwind the \emph{call stack} entail the task of restoring the register values for each \emph{call frame} and find the location of the call frame on the stack.
Each \emph{call frame} has a code location that represent where it stopped for any reason, the reason could be that a break point was hit.
It could also mean the location where it made a call from or 


The \emph{call stack} is made up of \emph{call frames} that each store information for a function scope, the information consist of variables, arguments and some register values that are saved on the stack.
Most of the information is is easy to access because it is one the stack thus it can just be read, the only problem is knowing what the values on the stack corresponds to which variable, argument or register.
There is also some of the register values that are hard coded by the compiler into the machine code thus making it hard to find those values.

