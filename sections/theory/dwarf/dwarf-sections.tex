% Primarily this section should be about scientific methods and theories you need to evaluate/compare/invent to solve your problems from 1.3.
% In some cases it may be ok to describe different technologies, but the purpose is to describe something and then draw a conclusion from that.
% Example, if you decide to discuss different databases, it may be for the purpose of selecting the best type for your implementation later on (based on for example data representation, scalability, speed, etc.).
% Optimally the problems in 1.3 are not solved by anyone else yet, in which case this section needs to describe how to solve them (new algorithms, mathematical approaches, etc.).
 
% This section can have a lot of subsections (3.1, 3.2, 3.3, etc).


% TODO: Explain DWARF Sections
The \emph{DWARF} format is divided into sections in the object file which all contain specific information, these sections use offsets to point to information in another section, see figure \ref{fig:dwarfsections}.
All of the sections can be different from depending on \emph{DWARF} versions and some doesn't exist in the older versions.
Thus these explanations only apply to \emph{DWARF} version $4$ and some of the older versions, checkout Appendix F in \cite{dwarf} for more information.

\begin{figure}[h]
    \centering
    \includegraphics[width=1.0\textwidth]{dwarf-sections.png}
    \label{fig:dwarfsections}
\end{figure}


Going in alphabetical order the first \emph{DWARF} section is called \emph{.debug\_abbrev}.
This section contain all of the abbreviation tables which can be used to find a specific die using the abbreviation.
These table entries contain information about the die tag, attributes and if it has children.
The library \emph{gimli-rs} simplifies the process of using this table and thus removes the need to know the detail of how to read it, but checkout section 7.5.3 in \cite{dwarf} to know more.


The \emph{DWARF} section \emph{\.debug\_aranges} is used to lookup which machine address corresponseds to which compilations unit.
This address information is stored in ranges where a compilation unit can have multiple ranges.
These ranges consists of a start address followed by lenght.
Thus to lookup the user only needs to check if the search addres is between the start addres and the start address plus the lengh.
To read more about this section checkout section 6.1.2 in \cite{dwarf}.


In the \emph{DWARF} section \emph{.debug\_frame} the infromation needed to virtualy unwind the call stack is kept.
Unwinding the call stack is complex and is hardware specific but the \emph{gimli-rs} library simplifies the process a lot.
This section is completly self-contained and is made up of two structures called \acrfull{cie} and \acrfull{fde}.
To learn more of this section checkout section 6.4.1 in \cite{dwarf}


Infomation about the source code are store in \glspl{die} which are low-level representation of the source code.
These have a tag that describes what it represents, an example tag is \emph{DW\_TAG\_variable} which means that the \gls{die} represents a variable from the source code.
All \glspl{die} are stored in tree structure that represents a compilation unit or a partial one.
These tree structures are structure like the source program and makes it possible to relate the source code to the machine code.
The section \emph{.debug\_info} contains a number of units that all have one of these \gls{die} trees and some more important debug information.
Thus this is one of the most important sections in \emph{DWARF} because it is used to understand the relation ship between the source code and the macine code.


The \emph{DWARF} section \emph{.debug\_line} holds the needed informtion to find the machine addresses that is generated from a certain line and column in the source file.
It is also used to store the source director, file name, line number and column.
Then the \glspl{die} will store pointers to the source location information in the section \emph{.debug\_line} enabeling the debuger to know the source location of a \gls{die}.
The section 6.2 in \cite{dwarf} explains in more detail how this informatiobn is stored in the \emph{.debug\_line} section.


\emph{.debug\_loc}
\emph{.debug\_macinfo}
\emph{.debug\_pubnames}
\emph{.debug\_pubtypes}
\emph{.debug\_ranges}
\emph{.debug\_str}
\emph{.debug\_type}

