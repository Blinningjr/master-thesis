% Primarily this section should be about scientific methods and theories you need to evaluate/compare/invent to solve your problems from 1.3.
% In some cases it may be ok to describe different technologies, but the purpose is to describe something and then draw a conclusion from that.
% Example, if you decide to discuss different databases, it may be for the purpose of selecting the best type for your implementation later on (based on for example data representation, scalability, speed, etc.).
% Optimally the problems in 1.3 are not solved by anyone else yet, in which case this section needs to describe how to solve them (new algorithms, mathematical approaches, etc.).
 
% This section can have a lot of subsections (3.1, 3.2, 3.3, etc).

Debugging refers to the process of finding and resolving errors, flaws, or faults in computer programs.
In computer science an error, flaw, or fault is often referred to as a software bug or just bug.
Bugs are the cause for software behaving in a unexpected way which leads to incorrect or unexpected results.
Most bugs arise from badly written code, lack of communication between the developers and lack of knowledge.


There are multiple ways to debug computer programs.
One of the ways is testing, where some input is sent into the code, and then the result is compared to the expected result that is known beforehand.
The amount of code being tested in a test can wary from just one function to the whole program.
Another way of debugging is to do a control flow analysis to see which order the instructions, statements, or function calls are done in.
There are a lot more ways to debug computer programs, but there is one that is most relevant to this thesis.
This last debugging technique requires a computer program called a debugger that can inspect what is happening in the program being debugged, it can also control the execution of the debugged program.



\subsubsection{Debugger}
\subimport{}{debuggers.tex}

