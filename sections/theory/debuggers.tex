% Primarily this section should be about scientific methods and theories you need to evaluate/compare/invent to solve your problems from 1.3.
% In some cases it may be ok to describe different technologies, but the purpose is to describe something and then draw a conclusion from that.
% Example, if you decide to discuss different databases, it may be for the purpose of selecting the best type for your implementation later on (based on for example data representation, scalability, speed, etc.).
% Optimally the problems in 1.3 are not solved by anyone else yet, in which case this section needs to describe how to solve them (new algorithms, mathematical approaches, etc.).
 
% This section can have a lot of subsections (3.1, 3.2, 3.3, etc).


A debugger is a computer program that is used for testing and debugging other computer programs.
The program that is being debugged is often referred to as the target program or just the target.
The two main functionalities of a debugger is firstly the ability to control the execution of the target program.
Secondly it is to translate the state of the target program into something that is more easily understandable.


Some of the most common ways a debugger can control a target program is starting, stopping, stepping, and resetting it.
Starting or continuing means to continue the execution of the target program.
Stopping the target program can often be done in two ways, the first is just to stop it where it is, the other way is to set a breakpoint.
A breakpoint is a point in the code that if reached will stop the target program immediately.
Stepping is the process of continuing the execution of the target program for only a moment, often just until the next source code line is reached.
Lastly resetting means that the target program will start execution from the start of the program.


Most debugger display the state of the target program relative to the source code.
This means that if the target program has stopped, most debuggers will translate the location in the machine code it stopped on into the location of the source instruction from where the machine code instruction was generated from.
They also often let the user set the breakpoint in the source code, and translate that to the closest machine code instruction.
Other features debuggers have is the ability to virtually unwind the call stack, evaluate variables, and to evaluate expression.
There are a lot more functionalities that a debugger can have, but these are some of the most common and used.

