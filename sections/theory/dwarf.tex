% Primarily this section should be about scientific methods and theories you need to evaluate/compare/invent to solve your problems from 1.3.
% In some cases it may be ok to describe different technologies, but the purpose is to describe something and then draw a conclusion from that.
% Example, if you decide to discuss different databases, it may be for the purpose of selecting the best type for your implementation later on (based on for example data representation, scalability, speed, etc.).
% Optimally the problems in 1.3 are not solved by anyone else yet, in which case this section needs to describe how to solve them (new algorithms, mathematical approaches, etc.).
 
% This section can have a lot of subsections (3.1, 3.2, 3.3, etc).


% Explain what this section will contain.
The \gls{DWARF} fromat is very complex which makes it hard to understand.
This section will go through how the fromat is structured and how the diffrent parts can be used to get debug information.
But it will not go trough every detail about the \gls{DWARF} format, because the \gls{DWARF} format has a great specification that goes into detail how it is structed and work.
Thus to read more about \gls{DWARF} checkout the specification \cite{dwarf}.


% Explain DWARF Sections
\subsubsection{Dwarf Sections}
\subimport{dwarf/}{dwarf-sections.tex}


% Explain DWARF Unit
\subsubsection{Dwarf Compilation Unit}
\subimport{dwarf/}{dwarf-unit.tex}


% TODO: Explain DWARF Die and Attributes
\subsubsection{Dwarf Debugging Information Entry}
\subimport{dwarf/}{dwarf-die.tex}


% TODO: Explain DWARF Attributes
\subsubsection{Dwarf Attribute}
\subimport{dwarf/}{dwarf-attribute.tex}


% TODO: Explain evaluation
\subsubsection{Evaluate Variable}
\label{sec:evaluate-variable}
\subimport{dwarf/}{evaluate.tex}


% TODO: Explain stacktrace
\subsubsection{Virtually Unwind Call Stack}
\subimport{dwarf/}{stacktrace.tex}

