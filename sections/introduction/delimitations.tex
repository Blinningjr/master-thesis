% Describe what is not covered in the thesis.
% Things you realize may have to be addressed to create a complete solution, but that would be too much work, or that may simply be out of the scope of your scientific area.


% Delimitations
% 	* Making changes to the compiler(llvm).
%	* Focusing on the bare minimum of debugger features.
%	* Making changes to DWARF standard.
%	* Limiting supported chips and platforms.
%	* Limiting to only support DWARF debug info and not the CodeView format. Source: https://llvm.org/docs/SourceLevelDebugging.html
%	* Limiting the depth of detail how dwarf works.
%	* Limiting it to a CLI and DAP server plus extension.



% Making changes to the compiler(rustc and llvm).
One of the main problems for getting a debugger to work for optimized code is getting the compiler to generate all the debug information needed.
In the case of the \emph{Rust} compiler \emph{rustc} it is the \emph{LLVM} library that mostly handles the debug information generation.
\emph{LLVM} is a very large project that many people are working on, and thus improving on \emph{LLVM} is out of scope for this thesis.
The same goes for improving \emph{rustc}.
%Thus this thesis is limited in solving this problem by using existing configurations for \emph{rustc} and \gls{LLVM}.


% Limiting to only support DWARF debug info and not the CodeView format. Source: https://llvm.org/docs/SourceLevelDebugging.html % TODO
The compiler backend \emph{LLVM} that \emph{rustc} uses supports two debugging file formats that holds all the debug information.
One of them is the format \gls{DWARF}, the other one is \emph{CodeView} which is developed by \emph{Microsoft}.
To make a debugger that supports both formats would be a lot of extra work that does not contribute to solving the main problem of this thesis.
Thus it has been decided to only support the \gls{DWARF} format because it has good documentation.


% Making changes to DWARF standard.
The scope of this thesis does also not include changing or adding to the \gls{DWARF} format.
The main reason is that it takes years for a new version of the standard to be released, and thus there is not enough time for this thesis to see, and realize that change or addition.
Another reason is that even if a new version of the \gls{DWARF} format could be released in the span of this thesis, it would take a lot of time before the \emph{Rust} compiler has been updated to use the new standard.
%Currently the newest \gls{DWARF} version is 5 but the \emph{Rust} compiler still uses the \gls{DWARF} 4 format.


% Will not explain every detail in dwarf.
The \gls{DWARF} format is very complex but provides good documentation.
Thus the explanation of \gls{DWARF} in section \ref{sec:dwarf} will not go into every detail of \gls{DWARF}.
Instead it will focus on explaining the minimum needed to understand the implementation of the debugger.
For further details we refer the reader to \cite{dwarf}.


%% Focusing on the bare minimum of debugger features
Today there are a lot of different debugging features that a debugger can have.
Many of them are advanced and complicated to implement, thus it is decided to limit the amount of features to the ones that are most important.
The following is the list of features the debugger is planed to have:

\begin{itemize}
  \item Controlling the debugging target by:
  \begin{itemize}
    \item Starting/Continuing execution.
    \item Stopping/Halting execution.
    \item Reset execution.
  \end{itemize}
  \item Set and remove breakpoints.
  \item Virtually unwind the call stack.
  \item Evaluate variables.
  \item Find source code location of functions and variables.
  \item A \acrfull{cli}.
  \item Support the \emph{Microsoft} \acrfull{dap}.
\end{itemize}


%Many of the debuggers today have a lot of functionality to help the user understand what is happening in the program that they are debugging.
%An example of these functionality are debuggers that support the ability to go backwards in the program.
%Functionalities like this are useful but dose not contributing much to the main problem of this thesis which is debugging optimized code.
%%Which is that most of the source code variables get optimized away and thus making it extremely hard to understand what is happening in the code.
%So to keep this thesis focused on the main problem the feature the debugger will have is restricted to virtually unwinding the stack, evaluating stack frames and there variables and lastly the ability to add and remove breakpoints.
%The debugger will also have the ability to control the program by stopping, continuing and stepping an instruction which are important debugger functionalities that does not require debug information.


% Limiting supported chips and platforms.
When debugging code on embedded systems the debugger needs to know a lot about the hardware, and some of these things differ depending on the hardware.
The following are some of the things it requires:

\begin{itemize}
  \item Number of registers.
  \item Which of the registers are the special ones, an example is the \gls{pc} register.
  \item The endianness of the memory.
  \item The machine code instruction set the \gls{mcu} is using/supports.
\end{itemize}

To support all the different \glspl{mcu} would be too much work for this thesis.
Thus the debugger is limited to work with the \emph{Nucleo-64 STM32F401} card because it is the one that is available, and it will only support the \emph{arm Thumb mode} instruction set..
%The debugger will be design to work with other similar \gls{mcu} but to test and grantee that it will work with them is to much work for this thesis.


%% Limiting it to a CLI and DAP server plus a VSCode extension.
%Another part of this thesis is the interaction between the user and the debugger.
%Existing debugger like \emph{GDB} both have a \emph{CLI} and a \acrfull{gui}, thus it is up to the user which one they want to use.
%From a usability perspective the debugger in this thesis should also have both of the option for the user to choose from.
%A \emph{CLI} is not that much work to implement but a \acrshort{gui} takes a lot of work to implement.
%Luckily \emph{Microsoft} has made a protocol for debuggers that specifies an adapter that handles the communication between the \acrshort{gui} and the debugger.
%This protocol is called \acrfull{dap}\cite{dap} and is used by \emph{VSCode}.
%Thus the scope of the debugger will include implementing the \acrshort{dap} protocol and a simple extension for \emph{VSCode}.


%% Limiting the depth of detail how dwarf works.
%The \gls{DWARF} format is very extensive and supports a lot of different program languages, the specifications for the different languages are a little different from each other.
%Because this thesis is about \emph{Rust} code the thesis will only go into detail in how to read the \gls{DWARF} format for \emph{Rust}.
%The specification for the \gls{DWARF} format is also very good at explaining how the information is structured.
%Thus a this thesis will not go into all the detail in how to read the \gls{DWARF} format instead it will focus on explaining how the information in the \gls{DWARF} format can be used in combination to get important information that the user wants.

% TODO: Add sources where needed in this subsection

