% Why is your work important? Typically this section relates strongly to the problems that you define in 1.3.
% The motivations can be societal, economical, or other things, but try and look past the task at hand and find out what impact your results will have.

% Motivations bullet points
%	* Optimized rust code is up to 100 faster.
% 	* Improving the state of debugging.
% 	* Contributing with a solution written in Rust.
% 	* Contributing a debug rust library / simplifying debugging.
% 	* Shinning a light on the obstacles.
%	* Companies spend lot of money on testing but not on verifying if there program is correct, which debugging is a important tool for verifying.


% Optimized rust code is up to 100 faster.
The main motivation is that optimized \emph{Rust} code can be up to 100 times faster then unoptimised code. % TODO: Add Source
That is a very large difference in speed compared to other languages like \emph{C} for example, were the optimized code is about 2-4 times faster then the unoptimized code. % TODO: Add Source
Thus for language like \emph{C} it is much more acceptable to not be able to debug optimized code because the different isn't that large compared to \emph{Rust}.
But in the case of \emph{Rust} code the different is so large that some programs are to slow to run without optimization.
This causes the problem that the code can't be debugged because debuggers don't work well on optimized code and unoptimized code is to slow to run.
Because of that there is a need for debuggers that can give enough information to debug optimized \emph{Rust} code.
Then there is also the argument that relying only on testing the optimized code instead of going through and checking that is is correct is bad.
The reason being that there can be extremely many paths that needs to be tested and it is sometimes not feasible to test all of them.
In these cases it is less costly to verify that the code is correct then to test every path.


% Improving the state of debugging.
Another large motivation for this thesis is that the most common way of debugging \emph{Rust} code on embedded systems is complicated for a beginner.
One of the reason being that it requires two programs being \emph{openocd} and \emph{gdb}, it also requires configuration files which takes some time to understand and configure.
This is not very accessible for people that have no experience with these programs and is unnecessarily complicated.
The ideal solution for accessibility would be to have a single program instead of two and that it requires less configuring, thus making it as easy as possible for a new person to debug there code.


% Contributing with a solution written in Rust.
Most of the debuggers used for \emph{Rust} code are written in other programming languages and there isn't a lot of debugging tools written in \emph{Rust} yet.
Thus one of the motivation for the thesis debugger to be written in \emph{Rust} is to contribute with a example of a debugger written in \emph{Rust} to the \emph{Rust} community.
This also relates back to improving debugging for optimized \emph{Rust} code because that is a large and hard problem that requires a lot of work to solve and to maintain the solution.
One of the most realistic way that will happen is if the \emph{Rust} community around debugging grows and more people contribute with there solutions and ideas.


% Contributing a debug rust library / simplifying debugging.
Another way this thesis contributes to the \emph{Rust} community is by making a library that simplifies the process of retrieving information form the debug information.
This is important because it makes retrieving debug information simpler for new developers to start contributing to the \emph{Rust} debug community.
Which will hopefully lead to better debugging for optimized \emph{Rust} code.


% Shinning a light on the obstacles. % TODO
% Companies spend lot of money on testing but not on verifying if there program is correct, which debugging is a important tool for verifying. %TODO


